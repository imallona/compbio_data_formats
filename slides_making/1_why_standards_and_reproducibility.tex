\documentclass{beamer}
\usetheme{Warsaw} 
\usepackage[utf8]{inputenc}
\usepackage[T1]{fontenc}
% \usepackage[spanish]{babel}
\usepackage{xcolor}
\usepackage{soul}

\usepackage{default}
\usepackage{listings}
% \usepackage{color}
\usepackage{graphicx}
\usepackage[normalem]{ulem}
% \definecolor{links}{HTML}{2A1B81}
\hypersetup{colorlinks,linkcolor=,urlcolor=blue}

% \usepackage{xcolor}
% \usepackage{soul}
% \sethlcolor{blue}
% \renewcommand{\texttt}[1]{\hl{\ttfamily #1}}


% \definecolor{Light}{gray}{.90}
% \sethlcolor{Light}

% \let\OldTexttt\texttt
% \renewcommand{\texttt}[1]{\OldTexttt{\hl{#1}}}% will affect all \texttt
% %\newcommand{\hltexttt}[1]{\texttt{\hl{#1}}}% comment above \renewcommand if want this


% \usepackage[dvipsnames,svgnames,x11names]{xcolor} % for lots of predefined color names
% \renewcommand\texttt[2][blue]{\textcolor{#1}{\texttt{#2}}}


\title{Why standards? Good practices in computational biology}
\institute{COST Project Epichembio - Introduction to NGS data analysis}
\author[izaskun.mallona@sib.swiss]{Izaskun Mallona}

\date{13th March 2019}

% \DeclareUnicodeCharacter{00A0}{ }

\begin{document}

\begin{frame}
  \titlepage
\end{frame}


\begin{frame}
  \frametitle{Acknowledgements}
  \begin{itemize}
  \item EU Horizon 2020 COST Project Epichembio 
  \item IJC Carreras Foundation
  \item Organizers: Sarah, Marguerite-Marie, David, Roberto
  \item SIB Swiss Institute of Bioinformatics and Univ. Zurich
  \end{itemize}
\end{frame}

% \section{Introduction}



\begin{frame}
  \frametitle{Talk typesetting}
  \begin{itemize}
  \item Commands/options are in \texttt{typewriter font}
  \item URLs are highlighted in \href{https://genome-euro.ucsc.edu/cgi-bin/hgGateway}{blue} (you should be able to browse the hyperlink)
  % \item Exercises are highlighted in {\color{red}red}
  \end{itemize}
\end{frame}


\begin{frame}
  \frametitle{Exercise: Web browsing the genome}
  \begin{itemize}
  \item \href{https://genome-euro.ucsc.edu/cgi-bin/hgGateway}{Launch the UCSC Genome Browser}
  \item  Specify \texttt{Human Assembly hg19}    
  \item Click \texttt{go}
  \end{itemize}

  By default, the Genome Browser will render a genomic window with many data layers on it. How are these data encoded?

  \begin{itemize}
  \item Click \texttt{UCSC Genes} from the \texttt{Genes and Gene Predictions} section under the main genomic window.
  \item  Click \texttt{View table schema} opens \href{https://genome-euro.ucsc.edu/cgi-bin/hgTables?db=hg19&hgta_group=genes&hgta_track=knownGene&hgta_table=knownGene&hgta_doSchema=describe+table+schema}{knownGene table schema}
    \end{itemize}

  \end{frame}

  \begin{frame}
    \frametitle{knownGene table schema}
\includegraphics[width=\linewidth]{figures/knownGene.png}

  \end{frame}


  \begin{frame}
    \frametitle{knownGene table schema}

So they are database entries with \textbf{chrom}, \textbf{start} and \textbf{end} features. This is the most standard data representation in genomics: data refering to genomic coordinates. Why?

  \end{frame}



  \begin{frame}
    \frametitle{Discussion}
    \begin{itemize}
    \item Which would be the most efficient file format to store data related to human genomes?

    \end{itemize}

  \end{frame}






% \section{Genomics data management}


\begin{frame}
  \frametitle{Commonly used formats}
  \begin{itemize}

  \item  Reference genomes

  \item  Fasta and FastQ (Unaligned sequences)
 
  \item SAM/BAM (Alignments)
 
  \item BED (Genomic ranges)
 
  \item GFF/GTF (Gene annotation)
 
  \item BEDgraphs (Genomic ranges)
 
  \item Wiggle files, BEDgraphs and BigWigs (Genomic scores).
 
  \item Indexed BEDgraphs/Wiggles

  \item VCFs (variants)

  \end{itemize}
\end{frame}

% \begin{frame}
% \frametitle{Selecting the optimal genomic data format}
% \begin{itemize}
% \item Will multiple types of features be stored in the same file?
% Does rich attribute information need to be saved in the file?
% Are features discontinuous?
% Is the computing environment limited for processing power or memory and/or is the feature annotation very large?

% \end{itemize}
% \end{frame}

\begin{frame}
  \frametitle{Reference genomes}
  \begin{itemize}
  \item Reference genomes describe the 'consensus' DNA sequence 
  \item (The human genome from whom? What does consensus mean?)
  \item Human variation aside, multiple assemblies have been released (i.e. hg18, hg19...)
  \end{itemize}
\end{frame}

\begin{frame}
  \frametitle{Sanger sequencing Nature 409, 863 (2001)}
\centering
\includegraphics[width=0.8\linewidth]{figures/10[1]_1038_nature01410-f1_full.jpg}
\end{frame}

\begin{frame}
  \frametitle{Hierarchical shotgun Nature 409, 863 (2001)}
\centering
\includegraphics[width=0.7\linewidth]{figures/v409_p861_409860a0-f2_FULL.jpg}
\end{frame}



\begin{frame}
  \frametitle{Reference genomes}
  GRCh stands for `Genome Reference Consortium`
  \begin{itemize}
  \item Human \href{https://www.ncbi.nlm.nih.gov/assembly/GCF_000001405.13/}{GRCh37} (hg19)
  \item Human \href{https://www.ncbi.nlm.nih.gov/assembly/GCF_000001405.38}{GRCh38}
  \item Mouse \href{https://www.ncbi.nlm.nih.gov/assembly/GCF_000001635.20/}{mm10}
  \item Mouse \href{https://www.ncbi.nlm.nih.gov/assembly/GCF_000001635.26}{GRCm38}

  \item Zebrafish, chicken and others: \url{https://www.ncbi.nlm.nih.gov/grc}{The Genome Reference consortium}

  \end{itemize}
\end{frame}



\begin{frame}
  \frametitle{Activity: sequence retrieval}
  \begin{itemize}
  \item Retrieve the sequence of the human chromosome 7...
  \item ... and make it traceable, replicable and reproducible
  \end{itemize}
\end{frame}


\begin{frame}
  \frametitle{Automation}
  \begin{itemize}
  \item Using a Web browser to retrieve genomic sequences is not efficient nor reproducible: programmatic alternatives exist
  \item Need of standardizing data analysis using reproducible workflows
    \begin{itemize}

    \item Scripts for data retrieval (bash, R, python...)
    \item Keeping track of data analysis steps and avoiding manual editing
    \item Data storage: standards (fasta, fastq, sam, vcf...)
    \end{itemize}
  \end{itemize}
\end{frame}




\begin{frame}
  \frametitle{Reproducibility}
  \begin{itemize}
  \item What do we mean by data science reproducibility? and replicability? and repeatability?
  \item In data science: avoid manual steps of data analysis using scripts plus version control systems
  \item Magics, blackboxes, untraceable stuff include:
    \begin{itemize}
    \item Spreadsheet manual editing
    \item Find-and-replace using a text editor
    \item Antyhing that involves mouse clicks without any log/macro
    \end{itemize}
  \end{itemize}

  % image gere
\end{frame}



% \begin{frame}
%   \frametitle{Reproducibility: reading}
%   \begin{itemize}
%   \item Paper: \url{https://www.nature.com/news/1-500-scientists-lift-the-lid-on-reproducibility-1.19970}
%     \end{itemize}
% \end{frame}

\begin{frame}
  \frametitle{Is there a reproducibility crisis?}
\centering
\includegraphics[width=0.8\textwidth]{figures/repro1.png}

{\tiny Baker M (2016) Is there a reproducibility crisis? Nature 533:452–454 9}
\end{frame}


\begin{frame}
  \frametitle{Is there a reproducibility crisis?}
\centering
\includegraphics[height=0.8\textheight]{figures/repro2.png}

{\tiny Baker M (2016) Is there a reproducibility crisis? Nature 533:452–454 9}
\end{frame}

\begin{frame}
  \frametitle{The causes}
\centering
\includegraphics[height=0.8\textheight]{figures/repro3.png}

{\tiny Baker M (2016) Is there a reproducibility crisis? Nature 533:452–454 9}
\end{frame}

\begin{frame}
  \frametitle{The alternatives}
\centering
\includegraphics[height=0.8\textheight]{figures/repro4.png}

{\tiny Baker M (2016) Is there a reproducibility crisis? Nature 533:452–454 9}
\end{frame}



\begin{frame}
  \frametitle{Good practices}
  \begin{itemize}
  \item Data
    \begin{enumerate}
    \item Using data standards
    \item Raw data availability
    \item Metadata
    \item Intermediate datasets availability (mid-processed, i.e. BED files)
    \end{enumerate}
  \item In data analysis
    \begin{enumerate}
    \item Scripting everything
    \item Version control
    \item Trace software versions/automate installs
    \item Release all code as supplementary information
    \end{enumerate}
  % \item What if we don't know how to program?
  %   \begin{itemize}
  %   \item Still, switching to command-line tools and keeping track of the commands used
  %   \item Request the source code when collaborating with computational biologists
  %   \end{itemize}
  \end{itemize}
\end{frame}



\begin{frame}
  \frametitle{Good practices}
 \begin{itemize}
  \item What if we don't know how to program?
    \begin{itemize}
    \item Still, switching to command-line tools and keeping track of the commands used
    \item Request the source code when collaborating with computational biologists
    \end{itemize}
  \end{itemize}
\end{frame}


\begin{frame}
  \frametitle{Cautionary note: science is science}
\centering
\includegraphics[height=0.7\textheight]{figures/repro_soccer.png}

{\tiny \url{https://fivethirtyeight.com/features/science-isnt-broken}}
\end{frame}


\begin{frame}
  \frametitle{The terminal}
  \begin{itemize}
    
  \item Simple command line interface 
  \item Present in MacOS and GNU/Linux
  \item Interprets the Unix shell language (commonly bash)
  \item Even though can be used in an interactive manner, commands can be written and stored as a script (=workflow, =pipeline)
  \item (A bash script is, probably, the simplest way of making a workflow repeatable)
  \end{itemize}

\end{frame}


\begin{frame}
  \frametitle{UNIX}
  \begin{itemize}
  \item Efficient
  \item Scalable
  \item Portable
  \item Open
  \end{itemize}
\end{frame}
% \begin{frame}
%   \frametitle{Basic shell commands}
% \centering
% \includegraphics[width=0.8\linewidth]{figures/bash.png}
% \end{frame}


\begin{frame}
  \frametitle{Unix philosophy}
According to Peter H. Salus in A Quarter-Century of Unix (1994):
  \begin{itemize}
  \item Write programs that do one thing and do it well
  \item Write programs to work together
  \item Write programs to handle text streams, because that is a universal interface
  \end{itemize}
\end{frame}


\begin{frame}
  \frametitle{Why in bioinformatics?}
  \begin{itemize}
  \item We interpret DNA, proteins as text; Unix is for text streams.
  \item Data are big (millions of lines of text, easily a couple of GB); spreadsheet software (Excel) cannot handle them.
  \item We need to keep track of our analysis for the sake of reproducibility: bash scripts.
  \end{itemize}
\end{frame}

\begin{frame}
  \frametitle{Opening a terminal in MacOS}
\centering
\includegraphics[width=0.8\linewidth]{figures/terminal.jpg}
\end{frame}

\begin{frame}
  \frametitle{Opening a terminal in GNU/Linux}
\centering
\includegraphics[width=0.8\linewidth]{figures/terminal_debian.png}
\end{frame}



% \begin{frame}[fragile]
%   \frametitle{Checking filetypes using \texttt{file}}
%   \begin{itemize}
%   \item To check file types we can use the \texttt{file} terminal command
%   \item Make sure the file you created is ASCII text!
%   \end{itemize}
% \begin{verbatim}
% [imallona@neutral tmp]$ file coordinate_file_formats.pdf 
% coordinate_file_formats.pdf: PDF document, version 1.5
% [imallona@neutral tmp]$ file coordinate_file_formats.tex
% coordinate_file_formats.tex: LaTeX 2e document, ASCII text
% \end{verbatim}


% \end{frame}




% \begin{frame}
% \frametitle{Exercise}
% \begin{itemize}
% \item Create a folder named \texttt{soft}
% \item Move into the folder
% \item Create a folder named \texttt{kent}
% \item Move into that folder
%   \end{itemize}
% \end{frame}

% \begin{frame}[fragile]
% \frametitle{Exercise}
% \begin{verbatim}
% cd ## goes to your home directory
% mkdir soft
% cd soft
% mkdir kent
% cd kent
% \end{verbatim}

% \end{frame}

% \begin{frame}[fragile]
%   \frametitle{The shell (Unix shell)}
%   \begin{itemize}
%   \item The Unix shell allows to save the sequential commands in a reproducible manner
%   \item Activity: run the tutorial by the Swiss Institute of Bioinformatics \url{https://edu.sib.swiss/pluginfile.php/2878/mod_resource/content/4/couselab-html/content.html}
%   \end{itemize}
% \end{frame}


  % \item Reading \url{https://blog.teamtreehouse.com/introduction-to-the-mac-os-x-command-line}




\begin{frame}
  \frametitle{A quick reminder on computer files}
  \begin{itemize}
  \item  Files are data representations stored in computers as arrays of bytes
  \item File type is defined by its bytes and not by the filename extension
  \item Files contain metadata
  \item Importantly, plain text files are composed by bytes mapped directly to ASCII characters
  \item Text editors (notepad, gedit, vim...) allow editing plain text files
  \item (text files can be read without proprietary software)
  \end{itemize}
\end{frame}


\begin{frame}[fragile]
  \frametitle{Questions}
  \begin{itemize}
  \item Sequences are often stored as FASTA, i.e.
\begin{verbatim}
>hg_19_chr7_short_version
tatatata
\end{verbatim}
  \item How to save this to a file to be a fasta?
    \begin{enumerate}
    \item Edit it with a text editor and save it as \texttt{test.fasta}
    \item Edit it with a text editor and save it as \texttt{test.fa}
    \item Edit it with a text editor and save it as \texttt{test.png}
    \item Edit it with a LibreOffice Writer and save it as an ODT file named \texttt{test.odt}
    \item Edit it with a LibreOffice Writer and save it as an ODT file named \texttt{test.fasta}
    \item Edit it with a R and \texttt{save()} it as an Rdata object \texttt{test.Rdata}
    \end{enumerate}
  \end{itemize}
\end{frame}


\begin{frame}[fragile]
  \frametitle{Questions}
  \begin{itemize}
  \item Sequences are often stored as FASTA, i.e.
\begin{verbatim}
>test
acgt
\end{verbatim}
  \item How to save this to a file to be a fasta?
    \begin{enumerate}
    \item[YES] Edit it with a text editor and save it as \texttt{test.fasta} 
    \item[YES] Edit it with a text editor and save it as \texttt{test.fa}
    \item[YES WTF] Edit it with a text editor and save it as \texttt{test.png}
    \item[NO] Edit it with a LibreOffice Writer and save it as an ODT file named \texttt{test.odt}
    \item[NO] Edit it with a LibreOffice Writer and save it as an ODT file named \texttt{test.fasta}
    \item[NO] Edit it with a R and \texttt{save()} it as an Rdata object \texttt{test.Rdata}
    \end{enumerate}
  \end{itemize}
\end{frame}


\begin{frame}
  \frametitle{Setting up the Mac text editor to save text}

  \begin{itemize}
  \item Create new file
  \item Go to \texttt{Format} and select \texttt{Make Plain Text}
  \item For saving, go to \texttt{File}, \texttt{Save As} and \texttt{Plain Text Encoding setting: Unicode (UTF-8).}
 
  \end{itemize}

\end{frame}

% \begin{frame}
%   \frametitle{}
% \includegraphics[width=\linewidth]{}
% \end{frame}


\begin{frame}
  \frametitle{Avoiding RTF in Mac: plain text}
\centering
\includegraphics[width=0.7\linewidth]{figures/mac-plain-txt.png}
\end{frame}


\begin{frame}
  \frametitle{Avoiding RTF in Mac: plain text}
\centering
\includegraphics[width=0.7\linewidth]{figures/mac-plain-txt2.png}
\end{frame}



% \begin{frame}
%   \frametitle{Turning off autospell check in Mac}
%   \begin{itemize}
%   \item UNIX commands are case sensitive
%   \item By default, TextEdit capitalizes/spell checks contents
%   \item How to disable this feature \url{http://osxdaily.com/2014/05/06/turn-off-autocorrect-pages-textedit-mac/}
%   \end{itemize}
% \end{frame}



\begin{frame}
  \frametitle{Shell scripting}
  \begin{itemize}
  \item Write on top the \href{https://en.wikipedia.org/wiki/Shebang_(Unix)}{shebang}
  \item Write the date and what's the script about, your name and date
  \item Tip: comment lines start with \#
  \item Introduce the commands (one line each)
  \item Save it as a text file!
  \item Commit the changes/backup the file
  \item Run the script to run the commands in batch typing \texttt{bash\ name\_of\_the\_script.sh}
  \end{itemize}

  
\end{frame}

% \newsavebox\myv

% \begin{frame}[fragile]
%   \frametitle{Software installs}
% \begin{lrbox}{\myv}\begin{minipage}{\textwidth}
% \begin{verbatim}
% curl http://hgdownload.soe.ucsc.edu/admin/exe/macOSX.x86_64/bedToBigBed \
%    > bedToBigBed
% curl http://gattaca.imppc.org/groups/maplab/imallona/teaching/hg19.genome \
%    > hg19.genome
% curl http://gattaca.imppc.org/groups/maplab/imallona/teaching/example.bed \
%    > example.bed

% ## adding exec permissions to the binary
% chmod a+x bedToBigBed

% ## running the actual command
% ./bedToBigBed example.bed hg19.genome out.bb
% \end{verbatim}
% \end{minipage}\end{lrbox}

% \resizebox{0.95\textwidth}{!}{\usebox\myv}

% \end{frame}


\begin{frame}
\frametitle{Reproducibility for software}
\begin{itemize}
\item UNIX solves the reproducibility, scalability and openness for data (text) streams, but extra software might be needed
\item The importance of software versioning for reproducibility: keeping track of the software installed
\item Using open source software (no blackboxes!)
\item Installs can be run command-line, so specific versions can be stored and included into the analysis script
\end{itemize}
  
\end{frame}


% \begin{frame}[fragile]
%   \frametitle{Software installs}
%   (The code is available at the \href{https://github.com/baudisgroup/BIO392-Github/blob/master/assets/imallona/3_exercises.md}{exercises file})

% \begin{tiny}
% \begin{verbatim}
% #!/bin/bash

% cd ~ # goes to your home folder 
% mkdir -p soft/kent
% cd soft/kent

% curl http://hgdownload.soe.ucsc.edu/admin/exe/macOSX.x86_64/bedToBigBed \
%    > bedToBigBed
% curl http://gattaca.imppc.org/groups/maplab/imallona/teaching/hg19.genome \
%    > hg19.genome
% curl http://gattaca.imppc.org/groups/maplab/imallona/teaching/example.bed \
%    > example.bed

% ## adding exec permissions to the binary
% chmod a+x bedToBigBed

% ## running the actual command
% ./bedToBigBed example.bed hg19.genome out.bb
% \end{verbatim}
% \end{tiny}

% \end{frame}


% \begin{frame}
%   \frametitle{Compiling bedtools}
%   \begin{itemize}
%   \item BEDtools as toolset
%   \item Activity: read paper \url{https://academic.oup.com/bioinformatics/article/26/6/841/244688}
%   \end{itemize}
% \end{frame}

% \begin{frame}
%   \frametitle{Compiling software in Unix}
%   \begin{itemize}
%   \item The importance of Makefiles
%   \item Activity: read \url{https://en.wikipedia.org/wiki/Makefile}
%   \end{itemize}
% \end{frame}




\begin{frame}[fragile]
  \frametitle{Compiling software -  bedtools}
  (The code is available at the exercises file)
Please note this is a bash script and specifies the exact software version to install

   \begin{small}
\begin{verbatim}
#!/bin/bash

cd # to your home directory
mkdir -p soft # creates a folder
cd soft # goes there

# the url is chopped into two pieces for readability
url_base=https://github.com/arq5x/bedtools2/releases/download/
curl -L "$url_base"/v2.25.0/bedtools-2.25.0.tar.gz \
  > bedtools-2.25.0.tar.gz

tar zxvf bedtools-2.25.0.tar.gz
cd bedtools2
make
\end{verbatim}
  \end{small}
\end{frame}


\begin{frame}
  \frametitle{Exercises}
  \begin{itemize}
  \item Run the exercises till number 4
  \end{itemize}
\end{frame}


\begin{frame}
  \frametitle{Next...}
  \begin{itemize}

  \item Fasta (Reference genomes)

  \item (Multi)fasta and FastQ (Unaligned sequences)
 
  \item SAM/BAM (Alignments)
 
  \item BED (Genomic ranges)
 
  \item GFF/GTF (Gene annotation)
 
  \item Wiggle files, BEDgraphs and BigWigs (Genomic scores).
 
  \item Indexed BEDgraphs/Wiggles

  \item VCFs (variants)

  \end{itemize}
\end{frame}



\end{document}


% \begin{frame}
%   \frametitle{Genomic standards}
% % lingua franca etc
%   \begin{itemize}

%   \item  Reference genomes

%   \item  Fasta and FastQ (Unaligned sequences)
 
%   \item SAM/BAM (Alignments)
 
%   \item BED (Genomic ranges)
 
%   \item GFF/GTF (Gene annotation)
 
%   \item BEDgraphs (Genomic ranges)
 
%   \item Wiggle files, BEDgraphs and BigWigs (Genomic scores).
 
%   \item Indexed BEDgraphs/Wiggles

%   \item VCFs (variants)

%   \end{itemize}
% \end{frame}



% \begin{frame}
%   \frametitle{Reference genomes: FASTA}
%   \begin{itemize}
%   \item A reference genome is a collection of contigs/scaffolds
%   \item  A contig is a stretch of DNA sequence encoded as A,G,C,T,N.	
%   \item  Typically comes in FASTA format.		
%   \item  ">" line contains the scaffold name
%   \item  Following lines contain the sequence (single line, 80 nt-column sized...)
%   \end{itemize}
% \end{frame}

% \begin{frame}[fragile]
%   \frametitle{Reference genomes: FASTA}
%   \begin{verbatim}
% >NC_009902.1 Babesia bovis T2Bo mitochondrion, complete genome
% TTTAAAAAAGTGTTAAAAACTTTATACATTAAAAAATTTAAACAAGTGATCATGTATAAAGTACACTTGT
% TACTGTGTAAATATCAAAAACAATTTAATTTCAAAATTTTTGAAATATGTTTTTTGTGTTGTGTTATAAA
% GTTTTTTTTCAAAATTATATATGTTTGCATTTGCTGGATATAGTTCGGTCTCTGCAAACCATAAAGTCAT
% CGGTATATCCTACATATGGCTTTCATATTGGTTTGGAGTTATTGGATTTTATATGAGTATTTTGATAAGA
% ACAGAATTGAGTATGAGTGGTTTAAAGATTATGACAATGGATACTCTTGAGATATACAATATGATGTTTT
%   \end{verbatim}
% \end{frame}

% \begin{frame}
%   \frametitle{Unaligned sequences (from sequencers): FASTQ}
%   \begin{itemize}
%   \item FASTQs stands for FASTA with Qualities
%   \item Plain text files with chunks of four lines:
%     \begin{itemize}
%     \item  @ identifier line
%     \item Sequence
%     \item  "+" 	
%     \item Quality scores (different encodings exist)
%     \end{itemize}
%   \end{itemize}
% \end{frame}

% \begin{frame}
%   \frametitle{Unaligned sequences (from sequencers): FASTQ}[fragile]
%   \begin{itemize}
%   \item FASTQ header
%     \begin{itemize}
%     \item 
%     \end{itemize}
%   \item Qualities
%     \begin{itemize}
%     \item 
%     \end{itemize}
%   \end{itemize}
% \end{frame}



% \begin{frame}[fragile]
%   \frametitle{Unaligned sequences (from sequencers): FASTQ (paired end)}
%   \begin{itemize}
%   \item  machine name HS2000-887\_89 - Machine name.		
%   \item  5 - Flowcell lane.		
%   \item  read in pair
%   \end{itemize}
% \end{frame}

% \begin{frame}[fragile]
%   \frametitle{FASTQ qualities}
%   \begin{itemize}
%   \item  phred score -log10 probability of sequence being wrong
%   \item  encoded in ASCII

%   \end{itemize}
% \end{frame}

% \begin{frame}
%   \frametitle{Alignment process}
% % figure here
% % gaps etc 
% \end{frame}

% \begin{frame}
%   \frametitle{Alignment file format}
%   \begin{itemize}
%   \item  SAM - Sequence Alignment Map.	
%   \item  Standard format for sequence data	
%   \item  Recognised by majority of software and browsers.
%   \item  Standards are important!

%   \end{itemize}
% \end{frame}


% \begin{frame}
%   \frametitle{SAM format}
%   \begin{itemize}
%   \item  SAM stands for Sequence Alignment Map.	
%   \item  Standard format for sequence data	
%   \item Recognised by majority of software and browsers.

%   \end{itemize}
% \end{frame}


% \begin{frame}
%   \frametitle{Aligned sequences}

%   \begin{itemize}
%   \item  Chromosome
%   \item  Locus (coordinate)
%   \item  CIGAR string, i.e.
%   \item 30M1D2M - 30 bases continuously match, 1 deletion from reference, 2 base match	
%   \item  Flags (\url{https://broadinstitute.github.io/picard/explain-flags.html})

%   \end{itemize}

% \end{frame}



% \begin{frame}
%   \frametitle{BAM format}
%  % Compilation into binary files

% % get header etc
% % etitle{}
%   \begin{itemize}
%   \item 
%   \end{itemize}
% \end{frame}

% \begin{frame}
%   \frametitle{Keep it simple: count and transform into BED files}
%   \begin{itemize}
%   \item BED (Browser Extensible Data)

%   \item BED3: 3 tab separated columns, chromosome (scaffold), start, end

%   \item BED6: BED3 plus name, score, strand


% % exercise: convert bed6 to bed3

%   \end{itemize}
% \end{frame}


% \begin{frame}
%   \frametitle{Comprexing and indexing}
%   \begin{itemize}
%   \item  SAM -> BAM (.bam and index file of .bai)		
%   \item  Wiggle and bedGraph -> bigWig		
%   \item  BED -> bigBed (.bb)		
%   \end{itemize}
% \end{frame}


% \begin{frame}
%   \frametitle{Data repositories}
%   \begin{itemize}
%   \item UCSC
%   \item Ensembl
% %  \item check the ones in github
%   \end{itemize}

% \end{frame}

% % add reading exercises etc


% % \begin{frame}
% %   \frametitle{Exercise: reference genomes}
% %   \begin{itemize}
% %   \item Retrieve the mitochondrial genome from UCSC
% %   \item 
% %   \end{itemize}
% % \end{frame}



% \end{document}






% \begin{frame}
%   \frametitle{}
% \includegraphics[width=\linewidth]{}
% \end{frame}

% \begin{frame}
%   \frametitle{}
%   \begin{itemize}
%   \item 
%   \end{itemize}
% \end{frame}
